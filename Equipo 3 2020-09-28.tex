%%%%%%%%%%%%%%%%%%%%%%%%%%%%%%%%%%%%%%%%%%%%%%%%%%%%%%%%%%%%%%%%%%%%%%%%%%%
%
% Plantilla para un articulo en LaTeX en español.
%
%%%%%%%%%%%%%%%%%%%%%%%%%%%%%%%%%%%%%%%%%%%%%%%%%%%%%%%%%%%%%%%%%%%%%%%%%%%

\documentclass{article}
\usepackage{listings}
\usepackage{enumerate}
% Esto es para poder escribir acentos directamente:
\usepackage[latin1]{inputenc}
\usepackage{graphicx}
% Esto es para que el LaTeX sepa que el texto esta en espa\~{n}ol:
\usepackage[spanish]{babel}
% Paquetes de la AMS:
\usepackage{amsmath, amsthm, amsfonts}
%Package de alineación
\usepackage[document]{ragged2e}
\spanishdecimal{.}

% Teoremas
%--------------------------------------------------------------------------
\newtheorem{thm}{Teorema}[section]
\newtheorem{cor}[thm]{Corolario}
\newtheorem{lem}[thm]{Lema}
\newtheorem{prop}[thm]{Proposici\'on}
\theoremstyle{definition}
\newtheorem{defn}[thm]{Definici\'on}
\theoremstyle{remark}
\newtheorem{rem}[thm]{Observaci\'on}

% Atajos.
% Se pueden definir comandos nuevos para acortar cosas que se usan
% frecuentemente. Como ejemplo, aqui se definen la R y la Z dobles que
% suelen representar a los conjuntos de numeros reales y enteros.
%--------------------------------------------------------------------------

\def\RR{\mathbb{R}}
\def\ZZ{\mathbb{Z}}

% De la misma forma se pueden definir comandos con argumentos. Por
% ejemplo, aqui definimos un comando para escribir el valor absoluto
% de algo mas facilmente.
%--------------------------------------------------------------------------
\newcommand{\abs}[1]{\left\vert#1\right\vert}

% Operadores.
% Los operadores nuevos deben definirse como tales para que aparezcan
% correctamente. Como ejemplo definimos en jacobiano:
%--------------------------------------------------------------------------
\DeclareMathOperator{\Jac}{Jac}

%--------------------------------------------------------------------------
\title{Actividades sesi\'on 28 de Septiembre\\
An\'alisis de Algoritmos III\\
Maestr\'ia en Ciencias de la Computaci\'on\\ 
Oto\~{n}o 2020\\
BUAP
}


\author{Equipo 3\\
  \small Erick Barrios Gonz\'alez\\
  \small Oscar Eduardo Gonz\'alez Ramos\\
  \small Oswaldo Jair Garc\'ia Franco
}

\begin{document}

\maketitle
\clearpage
%\abstract{Analisis de algoritmos}

\section{Ejercicios}
Resuelva las siguientes recurrencias exactamente con y sin manipulaci\'on algebraica:\\
\begin{enumerate}[1.]
\item  $$t_n = \left \{ 
\begin{matrix} 
n + 1 & \mbox{si n = }0\mbox{ o n = 1}\\ 
3t_{n-1}-2t_{n-2}+3 \cdot 2^{n-2} & \mbox{en}\mbox{ otro caso}\\
\end{matrix}
\right. $$\\

\begin{center}
\textbf{Con manipulaci\'on algebraica}\\
\end{center}

Reescribimos la recurrencia: \\
$$ t_{n}-3t_{n-1}-2t_{n-2} = 3 \cdot 2^{n-2} $$\\
La cual es de la forma de la Ecuaci\'on (1) con $b=2$ y $p(n)=3$ un polinomio de grado 0.\\
Reemplazamos n en la recurrencia por $n-1$ y multiplicar por $-2$\\
$$ -2t_{n-1} + 6t_{n-2} - 4t_{n-3} = - 3 \cdot 2^{n-2}	$$
As\'i obtenemos: \\
$$ t_{n} - 5t_{n-1} + 8t_{n-2} - 4t_{n-3} = 0$$
Su polinomio caracterist\'ico es:
$$ x^{3} - 5x^{2} + 8x - 4 = (x-1)(x-2)^{2}$$
Cuyas raices son $ r_{2} = 2 $ y $r_{1} = 2$ (ra\'ices m\'ultiples) y $r_{1}=1$\\

$$t_n= C_{1} + C_{2}^{n} + C_{3}n2^{n}$$
Si $C_{3}>0$ entonces $t_{n}$ $\epsilon$ $O(n2^{n})$\\
\clearpage

\begin{center}
\textbf{Sin manipulaci\'on algebraica}\\
\end{center}

Reescribimos la recurrencia: \\
$$ t_{n}-3t_{n-1}-2t_{n-2} = 3 \cdot 2^{n-2} $$\\
La cual es de la forma de la Ecuaci\'on (1) con $b=2$, $p(n)=3$ y $d=0$ un polinomio de grado 0.\\
As\'i su polinomio caracter\'istico es:\\
$$ (x-2)^{2}(x-1) $$
Cuyas raices son $ r_{2} = 2 $ y $r_{1} = 2$ (ra\'ices m\'ultiples) y $r_{1}=1$\\
$$t_n= C_{1} + C_{2}^{n} + C_{3}n2^{n}$$
Si $C_{3}>0$ entonces $t_{n}$ $\epsilon$ $O(n2^{n})$\\
Si sustituimos la Ecuaci\'on en la recurrencia original, obtenemos:\\
$$3 \cdot 2^{n-2} = (C_{1}+C_{2}^{n}+C_{3}n2^{n}) - 3(C_{1}+C_{2}^{n-1}+C_{3}(n-1)2^{n-1})$$
$$3 \cdot 2^{n-2} = \frac{C_{3}2^{n}}{2}$$
$$3 = \frac{C_{3}2^{n}}{2} \cdot \frac{1}{2^{n-2}}$$
$$3 = 2C_{3}$$ 
Por lo tanto\\
$$C_{3}=\frac{3}{2}$$
\clearpage



%Para el 2do ejercicio
\item  $$T(n)= \left \{ 
\begin{matrix} 
a & \mbox{si n = }0\mbox{ o n = 1}\\ 
T(n-1)+T(n-2)+C & \mbox{en}\mbox{ otro caso}\\
\end{matrix}
\right. $$\\

\begin{center}
\textbf{Con manipulaci\'on algebraica}\\
\end{center}

Reescribimos la recurrencia: \\
$$ T(n)-T(n-1)-T(n-2)-C $$\\
$$ T(n)-T(n-1)-T(n-2)=C $$\\
La cual es de la forma de la Ecuaci\'on (1) con $b=1$ y $p(n)=1$ un polinomio de grado 0.\\
Formando la nueva ecuaci\'on
$$ T(n)-T(n-1)-T(n-2)=C $$\\
Multiplicando por $-1$ y remplazando $n$ por $n-1$\\
$$ -T(n-1)+T(n-2)+T(n-3)=-C $$\\
As\'i obtenemos:\\
$ T(n)-T(n-1)-T(n-2)~~~~~~~~~~~~~~~=C $\\
$ ~~~~~~-T(n-1)+T(n-2)+T(n-3)=-C$\\
\rule{80mm}{0.1mm}\\
Sumando estas dos ecuaciones tenemos: \\
$$ T(n)-2T(n-1)+T(n-3)=0 $$\\
As\'i su polinomio caracter\'istico es:\\
$$ x^{3}-2x^{2}+1=0 $$
Cuyas raices son $ r_{1} = \frac{1+\sqrt{5}}{2} $, $ r_{2} = \frac{1-\sqrt{5}}{2}$, $r_{3}=1$ y $\theta=\frac{1+\sqrt{5}}{2}$\\
As\'i: \\
$$T(n) = C_{1}(\frac{1+\sqrt{5}}{2})^{n} + C_{2}(\frac{1-\sqrt{5}}{2})^{n} + C_{3} ~~~~~~~~(2)$$
\clearpage

\begin{center}
\textbf{Sin manipulaci\'on algebraica}\\
\end{center}

Sustituyendo (2) en la recurrencia original $T(n)-T(n-1)-T(n-2)=C$\\
$$T(n)=C_{1}\theta^{n} + C_{2}(\frac{1-\sqrt{5}}{2})^{n} + C_{3}$$
$$C=(C_{1}\theta^{n} + C_{2}(\frac{1-\sqrt{5}}{2})^{n} + C_{3})-(C_{1}\theta^{n-1} + C_{2}(\frac{1-\sqrt{5}}{2})^{n-1} + C_{3})-(C_{1}\theta^{n-2} + C_{2}(\frac{1-\sqrt{5}}{2})^{n-2} + C_{3})$$
$$C=C_{1}(\theta^{n}-\theta^{n-1}-\theta^{n-2})+C_{2}((\frac{1-\sqrt{5}}{2})^{n}-(\frac{1-\sqrt{5}}{2})^{n-1}-(\frac{1-\sqrt{5}}{2})^{n-2})+C_{3}(1-1-1)$$
\rule{130mm}{0.1mm}\\
$$C_{1}+C_{2}+C_{3}=0$$
$$C_{1}\theta + C_{2}(\frac{1-\sqrt{5}}{2}) + C_{3}=1$$
$$C_{1}\theta^{2} + C_{2}(\frac{1-\sqrt{5}}{2})^{2} + C_{3}=1$$
 $\theta=\frac{1+\sqrt{5}}{2}$
 
$$C_{1}=\frac{3\sqrt{5}}{10}+\frac{1}{2}$$
$$C_{2}=-\frac{3\sqrt{5}}{10}+\frac{1}{2}$$
$$C_{3} = 1 $$
\clearpage

%Para el 3er ejercicio
\item  $$T(n)= \left \{ 
\begin{matrix} 
a & \mbox{si n = }0\mbox{ o n = 1}\\ 
T(n-1)+T(n-2)+Cn & \mbox{en}\mbox{ otro caso}\\
\end{matrix}
\right. $$\\

\begin{center}
\textbf{Con manipulaci\'on algebraica}\\
\end{center}

Reescribimos la recurrencia: \\
$$ T(n)-T(n-1)-T(n-2)-Cn $$\\
Formando la nueva ecuaci\'on
$$ T(n)-T(n-1)-T(n-2)=Cn $$\\
Multiplicando por $-1$ y remplazando $n$ por $n-1$\\

$$ T(n)-T(n-1)-T(n-2)~~~~~~~~~~~~~~~~~~=Cn $$\\
$$ ~~~~~~~~~~~~~~-T(n-1)+T(n-2)+T(n-3)=-Cn + C $$\\
Sumando las ecuaciones tenemos:\\
$$ T(n)-2T(n-1)+T(n-3)= -C $$\\
Multiplicando por $-1$ y remplazando $n$ por $n-1$ en la nueva ecuaci\'on\\
$$ -T(n-1)+2T(n-2)+T(n-4)=-C $$\\
Sumando las ecuaciones tenemos:\\
$$ T(n)-3T(n-1)+2T(n-2)+T(n-3)-T(n-4)= 0 $$\\
As\'i su polinomio caracter\'istico es:\\
$$ x^{4}-3x^{3}+2x^{2}+x-1= (x-1)^{2}(x^{2}-x-1)$$
Cuyas raices son $r_{1}=1$, $r_{2}=1$ de multiplicidad 2, $r_{3}=\frac{1+\sqrt{5}}{2}$ y $r_{4}=\frac{1-\sqrt{5}}{2}$\\
As\'i: \\
$$T(n)= C_{1}+C_{2}+C_{3}\theta^{n}+C_{4}(\frac{1-\sqrt{5}}{2})^n$$
Sustituyendo en la recurrencia original\\
$$C_{1}+nC_{2}+C_{3}\theta^{n}+C_{4}(\frac{1-\sqrt{5}}{2})^n - C_{1}-(n-1)C_{2}-C_{3}\theta^{n-1}-C_{4}(\frac{1-\sqrt{5}}{2})^{n-1}-$$
$$C_{1}-(n-2)C_{2}-C_{3}\theta^{n-2}-C_{4}(\frac{1-\sqrt{5}}{2})^n$$
\clearpage

\begin{center}
\textbf{Sin manipulaci\'on algebraica}\\
\end{center}

$$T(n)-T(n-1)-T(n-2)=Cn$$\\
La cual es de la forma de la Ecuaci\'on (1) con $b=1$, $p(n)=n$ y $d=1$ un polinomio de grado 1.\\
As\'i su polinomio caracter\'istico es:\\
$$(x^{2}-x-1)(x-1)^{2}$$
Cuyas raices son $r_{1}=\frac{1+\sqrt{5}}{2}$ y $r_{2}=\frac{1-\sqrt{5}}{2}$ $r_{3}=1$, $r_{4}=1$ de multiplicidad 2, \\
Entonces \\
$$T(n)= C_{1}\Theta^{n}+C_{2}(\frac{1-\sqrt{5}}{2}^{n})+C_{3}+C_{4}n$$
\clearpage


%Para el 4to ejercicio
\item  $a_{n}=a_{n-1}+n+2,~~~~~a_{0}=0$

\begin{center}
\textbf{Con manipulaci\'on algebraica}\\
\end{center}

Reescribimos la recurrencia: \\
$$a_{n}-a_{n-1}=n+2$$
Multiplicando por $-1$ y remplazando $n$ por $n-1$\\
$$-a_{n-1}+a_{n-2}=-n-1$$

Sumando las ecuaciones tenemos:\\
$$a_{n}-a_{n-1}~~~~~~~~~~~=n+2$$
+
$$~~~~~~~-a_{n-1}+a_{n-2}=-n-1$$
\rule{130mm}{0.1mm}\\
$$a_{n}-2a_{n-1}+a_{n-2}=1$$
Multiplicando por $-1$ y remplazando $n$ por $n-1$ en la nueva ecuaci\'on\\
$$-a_{n-1}+2a_{n-2}-a_{n-3}=-1$$
Sumando las ecuaciones anteriotres tenemos:\\
$$a_{n}-2a_{n-1}+a_{n-2}~~~~~~~~~~=1$$
+
$$~~~~~~-a_{n-1}+2a_{n-2}-a_{n-3}=-1$$
\rule{130mm}{0.1mm}\\
$$a_{n}-3a_{n-1}+3a_{n-2}-a_{n-3}=0$$
As\'i su polinomio caracter\'istico es:\\
$$ x^{3}-3x^{2}+3x-1= (x-1)^{3}$$
Cuya ra\'iz es $r_{1}=1$ de multiplicidad 3\\
Por lo tanto todas las soluciones son de la forma:\\
$$a_{n}=C_{1}+C_{2}n+C_{3}n^{2}$$
\clearpage
La soluci\'on general se encuentra resolviendo el siguiente sistema de ecuaciones lineales.\\
$$C_{1}=0~~~~~~~~~~~~~~~~~~~~~n=0$$
$$C_{1}+C_{2}+C_{3}=1~~~~~~~n=1$$
$$C_{1}+2C_{2}+4C_{3}=2~~~~~n=2$$
Donde obtenemos que:\\
$$C_{1}=0, C_{2}=1,  C_{3}=0$$
\clearpage



\begin{center}
\textbf{Sin manipulaci\'on algebraica}\\
\end{center}

Reescribimos la recurrencia: \\
$$a_{n}-a_{n-1}=n+2$$
En este caso, $b=1$ y $p(n)=n$\\
As\'i su polinomio caracter\'istico es:\\
$$(x-1)^{3}(x-2)$$
Cuyas raices son $r_{1}=2$ y $r_{2}=1$ de multiplicidad 3 \\
Por lo tanto todas las soluciones son de la forma:\\
$$a_{n}=C_{1}2^{n}+C_{2}1^{n}+C_{3}n1^{n}+C_{4}n^{2}1^{n} $$
$$a_{n}=C_{1}2^{n}+C_{2}+C_{3}n+C_{4}n^{2}$$
Sustituyendo en la recurrencia original\\
$$n+2= a_{n}-a_{n-1}$$
$$n+2=(C_{1}2^{n}+C_{2}+C_{3}n+C_{4}n^{2})-(C_{1}2^{n-1}+C_{2}+C_{3}(n-1)+C_{4}(n^{2}-2_{n}+1))$$
$$n+2=C_{1}(2^{n} - 2^{n-1})+C_{3}(n-n+1)+C_{4}(n^{2}-2n+1) $$
$$n+2=C_{1}2^{n-2}+C_{3}+C_{4}(2^{n})-C_{4}$$
\clearpage



%Para el 5to ejercicio
\item  $a_{n}=a_{n-1}+7n,~~~~~a_{0}=0$

\begin{center}
\textbf{Con manipulaci\'on algebraica}\\
\end{center}

Reescribimos la recurrencia: \\
$$a_{n}-a_{n-1}=7n$$
Multiplicando por $-1$ y remplazando $n$ por $n-1$\\
$$-a_{n-1}+a_{n-2}=7n+7$$

Sumando las ecuaciones tenemos:\\
$$a_{n}-a_{n-1}~~~~~~~~~~=7n$$
+
$$~~~~~~~-a_{n-1}+a_{n-2}=7n+7$$
\rule{130mm}{0.1mm}\\
$$a_{n}-2a_{n-1}+a_{n-2}=7$$
Multiplicando por $-1$ y remplazando $n$ por $n-1$ en la nueva ecuaci\'on\\
$$-a_{n-1}+3a_{n-2}-a_{n-3}=-7$$
Sumando las ecuaciones anteriotres tenemos:\\
$$a_{n}-2a_{n-1}+a_{n-2}~~~~~~~~~~=7$$
+
$$~~~~~~-a_{n-1}+3a_{n-2}-a_{n-3}=-7$$
\rule{130mm}{0.1mm}\\
$$a_{n}-3a_{n-1}+3a_{n-2}-a_{n-3}=0$$
As\'i su polinomio caracter\'istico es:\\
$$ x^{3}-3x^{2}+3x-1= (x-1)^{3}$$
Cuya ra\'iz es $r_{1}=1$ de multiplicidad 3\\
Por lo tanto todas las soluciones son de la forma:\\
$$a_{n}=C_{1}+C_{2}n+C_{3}n^{2}$$
\clearpage
La soluci\'on general se encuentra resolviendo el siguiente sistema de ecuaciones lineales.\\
$$C_{1}=0~~~~~~~~~~~~~~~~~~~~~n=0$$
$$C_{1}+C_{2}+C_{3}=1~~~~~~~n=1$$
$$C_{1}+2C_{2}+4C_{3}=2~~~~~n=2$$
Donde obtenemos que:\\
$$C_{1}=0, C_{2}=1,  C_{3}=0$$

\clearpage
\begin{center}
\textbf{Sin manipulaci\'on algebraica}\\
\end{center}

Reescribimos la recurrencia: \\
$$a_{n}-a_{n-1}=7$$
En este caso, $b=7$ , $p(n)=n$ y d=1\\
As\'i su polinomio caracter\'istico es:\\
$$(x-1)(x-7)^{2}$$
Cuyas raices son $r_{1}=1$ , $r_{2}=7$ y $r_{3}=7$ de multiplicidad 2 \\
Por lo tanto todas las soluciones son de la forma:\\
$$a_{n}=C_{1}+C_{2}7^{n}+C_{3}n7^{n} $$
La soluci\'on general se encuentra resolviendo el siguiente sistema de ecuaciones lineales.\\
$$C_{1}+C_{2}+0=0~~~~~~~~~~~~~n=0$$
$$C_{1}+7C_{2}+7C_{3}=1~~~~~~~~n=1$$
$$C_{1}+49C_{2}+98C_{3}=2~~~~~n=2$$
Donde obtenemos que:\\
$$C_{1}=-\frac{1}{3}, C_{2}=\frac{1}{3},  C_{3}=-\frac{1}{7}$$
Por lo tanto\\
$$a_{n}=-\frac{1}{3}+(\frac{1}{3})7^{n}+n7^{n-1}$$
\clearpage




\end{enumerate}


\clearpage

\pagebreak 
% Bibliograf\'ia.
%-----------------------------------------------------------------
\begin{thebibliography}{99}
\bibitem{Mauricio2016} Lavalle Mart\'inez, Jos\'e de Jes\'us; La presentaci\'on sobre la tercera
parte de An\'alisis de algoritmos, An\'alisis y Dise\~{n}o de Algoritmos.
Buap, Oto\~{n}o 2020.

\bibitem {notas} Jim\'enez Salazer, H\'ector y Lavalle Mart\'inez, Jos\'e de Jes\'us; An\'alisis y Dise\~{n}o de Algoritmos. Traducci\'on de partes del libro Fundamentals of Algorithmics de Brassard and Bratley FCC - BUAP, 2020.





\end{thebibliography}

\end{document}